% !TEX TS-program = pdflatex
% !TEX encoding = UTF-8 Unicode

% This is a simple template for a LaTeX document using the "article" class.
% See "book", "report", "letter" for other types of document.

\documentclass[11pt]{article} % use larger type; default would be 10pt

\usepackage[utf8]{inputenc} % set input encoding (not needed with XeLaTeX)

%%% Examples of Article customizations
% These packages are optional, depending whether you want the features they provide.
% See the LaTeX Companion or other references for full information.

%%% PAGE DIMENSIONS
\usepackage{geometry} % to change the page dimensions
\geometry{a4paper} % or letterpaper (US) or a5paper or....
% \geometry{margin=2in} % for example, change the margins to 2 inches all round
% \geometry{landscape} % set up the page for landscape
%   read geometry.pdf for detailed page layout information

\usepackage{graphicx} % support the \includegraphics command and options

% \usepackage[parfill]{parskip} % Activate to begin paragraphs with an empty line rather than an indent

%%% PACKAGES
\usepackage{hyperref}
\usepackage{amsmath}
\usepackage{amssymb}
\usepackage{booktabs} % for much better looking tables
\usepackage{array} % for better arrays (eg matrices) in maths
\usepackage{paralist} % very flexible & customisable lists (eg. enumerate/itemize, etc.)
\usepackage{verbatim} % adds environment for commenting out blocks of text & for better verbatim
\usepackage{subfig} % make it possible to include more than one captioned figure/table in a single float
% These packages are all incorporated in the memoir class to one degree or another...

%%% HEADERS & FOOTERS
\usepackage{fancyhdr} % This should be set AFTER setting up the page geometry
\pagestyle{fancy} % options: empty , plain , fancy
\renewcommand{\headrulewidth}{0pt} % customise the layout...
\lhead{}\chead{}\rhead{}
\lfoot{}\cfoot{\thepage}\rfoot{}

%%% SECTION TITLE APPEARANCE
\usepackage{sectsty}
\allsectionsfont{\sffamily\mdseries\upshape} % (See the fntguide.pdf for font help)
% (This matches ConTeXt defaults)

%%% ToC (table of contents) APPEARANCE
\usepackage[nottoc,notlof,notlot]{tocbibind} % Put the bibliography in the ToC
\usepackage[titles,subfigure]{tocloft} % Alter the style of the Table of Contents
\renewcommand{\cftsecfont}{\rmfamily\mdseries\upshape}
\renewcommand{\cftsecpagefont}{\rmfamily\mdseries\upshape} % No bold!

%%% END Article customizations

%%% The "real" document content comes below...

\title{A Closed Form for the Cost of Loans}
\author{Galen Novello}
%\date{} % Activate to display a given date or no date (if empty),
         % otherwise the current date is printed 

\begin{document}
\maketitle

\begin{abstract}
In this paper I present a closed form for the cost of loans.  I first derived this formula some years ago, and thought little of it.  But in searching online I can not find it published anywhere, so I am now motivated to write it down.  In particular, a loan of $\$x $ with a fixed interest rate of $r$ and payment $y$ per interval will incur a total cost of $$C = y\log_{1+r} \left( \frac{y}{y-rx} \right)$$.
\end{abstract}

\section{A Quick Review of Interest}
\subsection{Simple Interest}
Suppose you invest \$1000 at an interest rate of 3\% per year.  The simplest way the interest can be applied is to wait until the end of each year and then add the interest to the value of the investment.  This problems is well understood and in this setting the value of the investment after $n$ years will be $1000(1.03)^{n}$.  In general, a principal investment of $\$A$ at a simple interest rate of $r$ per year will have the value $A(1+r)^n$ after $n$ years.

\subsection{Discreetly Compounded Interest}
Often times, interest is not applied in the simple fashion described above.  Instead, the interest rate is broken up and applied smaller pieces of interest are applied more frequently. For example, if a $3\%$ yearly interest rate is applied monthly, then $\frac{3}{12} = .25\%$ interest is applied per month. The value of the loan after $n$ years will then be $1000(1.025)^{12n}$.  This, again, is a very well understood problem and a principal investment of $\$A$ at an interest rate of $r$ per year compounded $k$ times per year will have the value $A(1+\frac{r}{k})^{nk}$ after $n$ years.

\subsection{Payment Intervals}
The reader may have noted that compounded interest can be rephrased in terms of simple interest by changing the unit of time.  Continuing from the example above a instead of thinking of a loan with a $3\%$ yearly interest rate compounded monthly, we could simply consider the loan to have a $.25\%$ monthly interest rate compounded simply and then express the value of the loan as $1000(1.025)^{N}$ where $N$ now represents the number of months since the initial investment instead of the number of years. In general, discreetly compounded interest can always be rephrased in terms of the interval over which interest is applied (henceforth to be referred to as the {\bf payment interval}),  and in this case a principal investment of $\$A$ at an interest $r$ per payment interval will have the value $A(1+r)^n$ after $n$ years.

\section{Paying off the Loan}
Suppose, now, that a loan of $\$x$ is taken with an interest rate of $r$ per payment interval and a payment of $y$. The the value of the loan after the first payment interval will then be 
$$x(1+r) - y.$$
The second application of interest will apply to this new balance, so after 2 payment intervals the value will be 
$$(x(1+r)-y)(1+r) - y = x(1+r)^{2} - y(1 + (1+r)).$$
Again, they 3rd application of interest must apply to this balance, so after 3 payment intervals the value will be
$$\left(x(1+r)^{2} - (y + y(1+r))\right)(1+r) -y =$$
$$ x(1+r)^{3} - y\left(1 + (1+r) + (1+r)^{2}\right)$$
The pattern is clear:  after $n$ payment intervals, the value of the loan will be 
$$x(1+r)^n - y \sum_{k=0}^{n-1} (1+r)^{k}.$$

\section{Determining Time to pay off the loan}
When the loan is paid off, the value of the loan is zero. To find the number of payment intervals required to pay off the loan we simply must solve the equation

$$x(1+r)^n - y \sum_{k=0}^{n-1} (1+r)^{k} = 0$$
or, equivalently, 
$$x(1+r)^n = y \sum_{k=0}^{n-1} (1+r)^{k}.$$

The sum on the right hand side is geometric, so the well formula for the partial sum of a geometric series:
$$\sum_{k=0}^{n-1} a^{n} = \frac{1-a^{n}}{1-a}$$ can be employed and our equation can be re-written
$$x(1+r)^n = y \left(\frac {1 - (1+r)^{n}}{1 - (1+r)} \right).$$
$$x(1+r)^n + \frac{y}{-r}(1+r)^{n} = \frac{y}{-r}$$
$$ (1+r)^{n} (x - \frac{y}{r}) = \frac{-y}{r}$$
$$n = \log_{1+r} \frac{\frac{-y}{r}}{x - \frac{y}{r}}$$
$$n = \log_{1+r} \frac{y}{y - rx}$$

Hence, the number of payment intervals it will take to pay off the loan is $$n = \log_{1+r} \frac{y}{y - rx}$$.

\section{Determining the Total Cost}
Given the formula for the number of payment intervals, the total cost of the loan can be recovered by noting that a payment of $\$y$ is made each payment interval, so the total cost of the loan is
$$C = yn = y\log_{1+r} \frac{y}{y - rx}$$

\section{Determining payments for time specific loans}
Many loans are set up to be paid off over a specific time period and/or number of payment intervals. If we are interested in what payments will lead to specific number of payment intervals we can re-work the equation for the number of payment intervals:

$$n = \log_{1+r} \frac{y}{y - rx}$$
$$(1+r)^{n} = \frac{y}{y-rx}$$
$$(1+r)^{n}(y-rx) = y$$
$$y( (1+r)^{n} - 1) = rx(1+r)^{n}$$
$$y = \frac{rx(1+r)^{n}}{(1+r)^{n} - 1}$$.

This equation can then be normalized by dividing by the loan amount, to be phrased in terms of payment.  In particular to make a loan with interest rate $r$ per interval have a lifetime of $n$ payment intervals the payment per interval should be $p$ percent of the principal borrowed where 
$$p = \frac{r(1+r)^{n}}{(1+r)^{n} - 1}$$


\end{document}
